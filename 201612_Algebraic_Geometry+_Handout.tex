%!TeX spellcheck = de_DE
%!TeX encoding = utf8
\documentclass{article}
%preamble starts here
\usepackage[left=3cm,right=3cm]{geometry}
\usepackage[T1]{fontenc}
\usepackage{lmodern}
\usepackage{standalone}
\usepackage{standalone}
\usepackage{xspace}
\usepackage{xcolor}
\usepackage{url}

\input{./Packages_tikz}
\usepackage{amsmath}
\usepackage{amssymb}
\usepackage{amsthm}
\usepackage{nicefrac}

\newtheorem{thm}{Theorem}
%\newcounter{thm}
\newtheorem{satz}[thm]{Satz} 
\newtheorem{propdef}[thm]{Proposition-Definition}
\theoremstyle{definition}
\newtheorem{defn}{Definition}
\newtheorem{nota}{Notation}
\newtheorem{algo}{Algorithmus}
%\newtheorem{defi}[thm]{Definition} 
\newtheorem{lem}[thm]{Lemma} 
\newtheorem{beh}[thm]{Behauptung} 
\newtheorem{bsp}[thm]{Beispiel} 

%mathoperators
\DeclareMathOperator{\im}{im}
\DeclareMathOperator{\coker}{coker}
\DeclareMathOperator{\ini}{in}
\DeclareMathOperator{\GCD}{ggT}
\DeclareMathOperator{\LCM}{kgV}

\input{./Packages_bibtex}
\input{./Abbreviations}
\input{./Abbreviations_math}
\usepackage[ngerman]{babel}
\renewcommand{\id}{\ensuremath{Id}}
\renewcommand{\familydefault}{\rmdefault}

%documentspecific
\title{Freie Aufloesungen und das Syszygien Theorem}







\begin{document}
\begin{defn}[Complex of \(R\)-modules \cite{Eis1}{1.10} ]
	A \bf{complex} of \(R\)-Modules is a sequence of modules \( F_{i} \)
	and maps \( F_{i} to F_{i-1} \) such that the compositions \( F_{i+1} \to F_{i} \to F_{i-1} \) are all zero.
	The \bf{homology} of this complex at \( F_{i} \) is the module 
	\[
		        \ker \left( F_{i} \to F_{i-1} \right)   \/ \im \left( F_{i+1} \to F_{i} \right)
		\]
	A \bf{free resolution} of an \( R\)-module \( M \)
	is a complex
	\[
	        \mathcal{F}: \hdots \to F_{n} \overset{\to}{\phi_{n}} 
	        \hdots \to F_{1} \overset{\to}{\phi_{1}} F_{0}
	\]
	of free \(R\)-Modules such that \( \coker \phi_{1} = M \) 
	and \( \mathcal{F} \)  is exact 
	(sometimes we add `` \( \to 0 \) '' to the right of \(\mathcal{F}\) 
	and then insist that \(\mathcal{F}\) be exact except at \( F_{0} \) ).
	We shall sometimes abuse this notation and say that an exact sequence
	\[
	        \mathcal{F}: \hdots \to F_{n} \overset{\to}{\phi_{n}} 
	        \hdots \to F_{1} \overset{\to}{\phi_{1}} F_{0}
	        \to M \to 0
	\]
	is a resolution of \( M \).
	The image of the map \( \phi_i \) is called the ith syzygy module of \(M \).
	A resolution \( \mathcal{F }\) is a \bf{graded free resolution } 
	if \( R \) is a graded ring, 
	the \(  F_{i}\) are graded free modules, 
	and the maps are homogeneous maps of degree 0.
	Of course only graded modules can have graded free resolutions.
	If for some \( n < \inf \) we have \( F_{n+1}=0 \),
	but \( F_{i} \neq 0 \forall 0 \le i \le n \), then we shall say that
	\( \mathcal{F}\) is a \bf{finite resolution of length} \( n\).
\end{defn}


\input{./bibtex}

\end{document}

