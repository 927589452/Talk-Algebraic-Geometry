%!TeX spellcheck = de_DE
%!TeX encoding = utf8

\documentclass{article}
%preamble starts here
\usepackage[left=3cm,right=3cm]{geometry}
\usepackage[T1]{fontenc}
\usepackage{lmodern}
\usepackage{standalone}
\usepackage[utf8]{inputenc} 
\usepackage{standalone}
\usepackage{xspace}
\usepackage{xcolor}
\usepackage{url}
\usepackage{hyperref}

\usepackage{tikz}
\usepackage{tikz-cd}
\usetikzlibrary{3d}
\usetikzlibrary{calc}
\usetikzlibrary{arrows}
\usetikzlibrary{babel}

\usepackage{amsmath}
\usepackage{amssymb}
\usepackage{nicefrac}

\newtheorem{satz}{Satz} 
\newtheorem{defi}{Definition} 
\newtheorem{lem}{Lemma} 
\newtheorem{beh}{Behauptung} 
\newtheorem{bsp}{Beispiel} 


\usepackage{cite}

%provides general Abbreviations
\providecommand{\bzgl}{bez\"uglich\xspace}
%Provides a list of shortcuts

%Shorthands
\providecommand{\Iso}{Isomorphismus\xspace}
\providecommand{\Ison}{Isomorphismen\xspace}
\providecommand{\Mor}{Morphismus\xspace}
\providecommand{\Morn}{Morphismen\xspace}
\providecommand{\Abb}{Abbildung\xspace}
\providecommand{\Abbn}{Abbildungen\xspace}
%Symbols
\providecommand{\Hom}{\ensuremath{ \mathbf{Hom } }}
\providecommand{\id}{\ensuremath{ \mathbb{id} }}

\providecommand{\Ob}{\ensuremath{  \mathbf{Ob} }}
%FRAKS
\providecommand{\FA}{\ensuremath{\mathfrak{A}}}
\providecommand{\FB}{\ensuremath{\mathfrak{B}}}
\providecommand{\FC}{\ensuremath{\mathfrak{C}}}
\providecommand{\FD}{\ensuremath{\mathfrak{D}}}
\providecommand{\FE}{\ensuremath{\mathfrak{E}}}
\providecommand{\FF}{\ensuremath{\mathfrak{F}}}
\providecommand{\FG}{\ensuremath{\mathfrak{G}}}
\providecommand{\FH}{\ensuremath{\mathfrak{H}}}
\providecommand{\FI}{\ensuremath{\mathfrak{I}}}
\providecommand{\FJ}{\ensuremath{\mathfrak{J}}}
\providecommand{\FK}{\ensuremath{\mathfrak{K}}}
\providecommand{\FL}{\ensuremath{\mathfrak{L}}}
\providecommand{\FM}{\ensuremath{\mathfrak{M}}}
\providecommand{\FN}{\ensuremath{\mathfrak{N}}}
\providecommand{\FO}{\ensuremath{\mathfrak{O}}}
\providecommand{\FP}{\ensuremath{\mathfrak{P}}}
\providecommand{\FQ}{\ensuremath{\mathfrak{Q}}}
\providecommand{\FR}{\ensuremath{\mathfrak{R}}}
\providecommand{\FS}{\ensuremath{\mathfrak{S}}}
\providecommand{\FT}{\ensuremath{\mathfrak{T}}}
\providecommand{\FU}{\ensuremath{\mathfrak{U}}}
\providecommand{\FV}{\ensuremath{\mathfrak{V}}}
\providecommand{\FW}{\ensuremath{\mathfrak{W}}}
\providecommand{\FX}{\ensuremath{\mathfrak{X}}}
\providecommand{\FY}{\ensuremath{\mathfrak{Y}}}
\providecommand{\FZ}{\ensuremath{\mathfrak{Z}}}
\providecommand{\Fa}{\ensuremath{\mathfrak{a}}}
\providecommand{\Fa}{\ensuremath{\mathfrak{a}}}
\providecommand{\Fb}{\ensuremath{\mathfrak{b}}}
\providecommand{\Fc}{\ensuremath{\mathfrak{c}}}
\providecommand{\Fd}{\ensuremath{\mathfrak{d}}}
\providecommand{\Fe}{\ensuremath{\mathfrak{e}}}
\providecommand{\Ff}{\ensuremath{\mathfrak{f}}}
\providecommand{\Fg}{\ensuremath{\mathfrak{g}}}
\providecommand{\Fh}{\ensuremath{\mathfrak{h}}}
\providecommand{\Fi}{\ensuremath{\mathfrak{i}}}
\providecommand{\Fj}{\ensuremath{\mathfrak{j}}}
\providecommand{\Fk}{\ensuremath{\mathfrak{k}}}
\providecommand{\Fl}{\ensuremath{\mathfrak{l}}}
\providecommand{\Fm}{\ensuremath{\mathfrak{m}}}
\providecommand{\Fn}{\ensuremath{\mathfrak{n}}}
\providecommand{\Fo}{\ensuremath{\mathfrak{o}}}
\providecommand{\Fp}{\ensuremath{\mathfrak{p}}}
\providecommand{\Fq}{\ensuremath{\mathfrak{q}}}
\providecommand{\Fr}{\ensuremath{\mathfrak{r}}}
\providecommand{\Fs}{\ensuremath{\mathfrak{s}}}
\providecommand{\Ft}{\ensuremath{\mathfrak{t}}}
\providecommand{\Fu}{\ensuremath{\mathfrak{u}}}
\providecommand{\Fv}{\ensuremath{\mathfrak{v}}}
\providecommand{\Fw}{\ensuremath{\mathfrak{w}}}
\providecommand{\Fx}{\ensuremath{\mathfrak{x}}}
\providecommand{\Fy}{\ensuremath{\mathfrak{y}}}
\providecommand{\Fz}{\ensuremath{\mathfrak{z}}}
%DOUBLELINE
\providecommand{\Ba}{\ensuremath{\mathbb{a}}}
\providecommand{\Bb}{\ensuremath{\mathbb{b}}}
\providecommand{\Bc}{\ensuremath{\mathbb{c}}}
\providecommand{\Bd}{\ensuremath{\mathbb{d}}}
\providecommand{\Be}{\ensuremath{\mathbb{e}}}
\providecommand{\Bf}{\ensuremath{\mathbb{f}}}
\providecommand{\Bg}{\ensuremath{\mathbb{g}}}
\providecommand{\Bh}{\ensuremath{\mathbb{h}}}
\providecommand{\Bi}{\ensuremath{\mathbb{i}}}
\providecommand{\Bj}{\ensuremath{\mathbb{j}}}
\providecommand{\Bk}{\ensuremath{\mathbb{k}}}
\providecommand{\Bl}{\ensuremath{\mathbb{l}}}
\providecommand{\Bm}{\ensuremath{\mathbb{m}}}
\providecommand{\Bn}{\ensuremath{\mathbb{n}}}
\providecommand{\Bo}{\ensuremath{\mathbb{o}}}
\providecommand{\Bp}{\ensuremath{\mathbb{p}}}
\providecommand{\Bq}{\ensuremath{\mathbb{q}}}
\providecommand{\Br}{\ensuremath{\mathbb{r}}}
\providecommand{\Bs}{\ensuremath{\mathbb{s}}}
\providecommand{\Bt}{\ensuremath{\mathbb{t}}}
\providecommand{\Bu}{\ensuremath{\mathbb{u}}}
\providecommand{\Bv}{\ensuremath{\mathbb{v}}}
\providecommand{\Bw}{\ensuremath{\mathbb{w}}}
\providecommand{\Bx}{\ensuremath{\mathbb{x}}}
\providecommand{\By}{\ensuremath{\mathbb{y}}}
\providecommand{\Bz}{\ensuremath{\mathbb{z}}}
\providecommand{\BA}{\ensuremath{\mathbb{A}}}
\providecommand{\BB}{\ensuremath{\mathbb{B}}}
\providecommand{\BC}{\ensuremath{\mathbb{C}}}
\providecommand{\BD}{\ensuremath{\mathbb{D}}}
\providecommand{\BE}{\ensuremath{\mathbb{E}}}
\providecommand{\BF}{\ensuremath{\mathbb{F}}}
\providecommand{\BG}{\ensuremath{\mathbb{G}}}
\providecommand{\BH}{\ensuremath{\mathbb{H}}}
\providecommand{\BI}{\ensuremath{\mathbb{I}}}
\providecommand{\BJ}{\ensuremath{\mathbb{J}}}
\providecommand{\BK}{\ensuremath{\mathbb{K}}}
\providecommand{\BL}{\ensuremath{\mathbb{L}}}
\providecommand{\BM}{\ensuremath{\mathbb{M}}}
\providecommand{\BN}{\ensuremath{\mathbb{N}}}
\providecommand{\BO}{\ensuremath{\mathbb{O}}}
\providecommand{\BP}{\ensuremath{\mathbb{P}}}
\providecommand{\BQ}{\ensuremath{\mathbb{Q}}}
\providecommand{\BR}{\ensuremath{\mathbb{R}}}
\providecommand{\BS}{\ensuremath{\mathbb{S}}}
\providecommand{\BT}{\ensuremath{\mathbb{T}}}
\providecommand{\BU}{\ensuremath{\mathbb{U}}}
\providecommand{\BV}{\ensuremath{\mathbb{V}}}
\providecommand{\BW}{\ensuremath{\mathbb{W}}}
\providecommand{\BX}{\ensuremath{\mathbb{X}}}
\providecommand{\BY}{\ensuremath{\mathbb{Y}}}
\providecommand{\BZ}{\ensuremath{\mathbb{Z}}}
%Provides a list of shortcuts
\usepackage{xspace}
\providecommand{\catname}[1]{\ensuremath{\mathbf{#1}}\xspace}
\providecommand{\Par}{\catname{Par}}
\providecommand{\Set}{\catname{Set}}
\providecommand{\CatC}{\catname{C}}
\providecommand{\CatVect}{\catname{Vect}}
\providecommand{\CatD}{\catname{D}}

%Shorthands
\providecommand{\Iso}{Isomorphismus\xspace}
\providecommand{\Ison}{Isomorphismen\xspace}
\providecommand{\Mor}{Morphismus\xspace}
\providecommand{\Morn}{Morphismen\xspace}
\providecommand{\Abb}{Abbildung\xspace}
\providecommand{\Abbn}{Abbildungen\xspace}
%Symbols
\providecommand{\Hom}{\ensuremath{ \mathbf{Hom } }}
\providecommand{\id}{\ensuremath{ \mathbb{id} }}

\providecommand{\Ob}{\ensuremath{  \mathbf{Ob} }}
%FRAKS
\providecommand{\FA}{\ensuremath{\mathfrak{A}}}
\providecommand{\FB}{\ensuremath{\mathfrak{B}}}
\providecommand{\FC}{\ensuremath{\mathfrak{C}}}
\providecommand{\FD}{\ensuremath{\mathfrak{D}}}
\providecommand{\FE}{\ensuremath{\mathfrak{E}}}
\providecommand{\FF}{\ensuremath{\mathfrak{F}}}
\providecommand{\FG}{\ensuremath{\mathfrak{G}}}
\providecommand{\FH}{\ensuremath{\mathfrak{H}}}
\providecommand{\FI}{\ensuremath{\mathfrak{I}}}
\providecommand{\FJ}{\ensuremath{\mathfrak{J}}}
\providecommand{\FK}{\ensuremath{\mathfrak{K}}}
\providecommand{\FL}{\ensuremath{\mathfrak{L}}}
\providecommand{\FM}{\ensuremath{\mathfrak{M}}}
\providecommand{\FN}{\ensuremath{\mathfrak{N}}}
\providecommand{\FO}{\ensuremath{\mathfrak{O}}}
\providecommand{\FP}{\ensuremath{\mathfrak{P}}}
\providecommand{\FQ}{\ensuremath{\mathfrak{Q}}}
\providecommand{\FR}{\ensuremath{\mathfrak{R}}}
\providecommand{\FS}{\ensuremath{\mathfrak{S}}}
\providecommand{\FT}{\ensuremath{\mathfrak{T}}}
\providecommand{\FU}{\ensuremath{\mathfrak{U}}}
\providecommand{\FV}{\ensuremath{\mathfrak{V}}}
\providecommand{\FW}{\ensuremath{\mathfrak{W}}}
\providecommand{\FX}{\ensuremath{\mathfrak{X}}}
\providecommand{\FY}{\ensuremath{\mathfrak{Y}}}
\providecommand{\FZ}{\ensuremath{\mathfrak{Z}}}
\providecommand{\Fa}{\ensuremath{\mathfrak{a}}}
\providecommand{\Fa}{\ensuremath{\mathfrak{a}}}
\providecommand{\Fb}{\ensuremath{\mathfrak{b}}}
\providecommand{\Fc}{\ensuremath{\mathfrak{c}}}
\providecommand{\Fd}{\ensuremath{\mathfrak{d}}}
\providecommand{\Fe}{\ensuremath{\mathfrak{e}}}
\providecommand{\Ff}{\ensuremath{\mathfrak{f}}}
\providecommand{\Fg}{\ensuremath{\mathfrak{g}}}
\providecommand{\Fh}{\ensuremath{\mathfrak{h}}}
\providecommand{\Fi}{\ensuremath{\mathfrak{i}}}
\providecommand{\Fj}{\ensuremath{\mathfrak{j}}}
\providecommand{\Fk}{\ensuremath{\mathfrak{k}}}
\providecommand{\Fl}{\ensuremath{\mathfrak{l}}}
\providecommand{\Fm}{\ensuremath{\mathfrak{m}}}
\providecommand{\Fn}{\ensuremath{\mathfrak{n}}}
\providecommand{\Fo}{\ensuremath{\mathfrak{o}}}
\providecommand{\Fp}{\ensuremath{\mathfrak{p}}}
\providecommand{\Fq}{\ensuremath{\mathfrak{q}}}
\providecommand{\Fr}{\ensuremath{\mathfrak{r}}}
\providecommand{\Fs}{\ensuremath{\mathfrak{s}}}
\providecommand{\Ft}{\ensuremath{\mathfrak{t}}}
\providecommand{\Fu}{\ensuremath{\mathfrak{u}}}
\providecommand{\Fv}{\ensuremath{\mathfrak{v}}}
\providecommand{\Fw}{\ensuremath{\mathfrak{w}}}
\providecommand{\Fx}{\ensuremath{\mathfrak{x}}}
\providecommand{\Fy}{\ensuremath{\mathfrak{y}}}
\providecommand{\Fz}{\ensuremath{\mathfrak{z}}}
%DOUBLELINE
\providecommand{\Ba}{\ensuremath{\mathbb{a}}}
\providecommand{\Bb}{\ensuremath{\mathbb{b}}}
\providecommand{\Bc}{\ensuremath{\mathbb{c}}}
\providecommand{\Bd}{\ensuremath{\mathbb{d}}}
\providecommand{\Be}{\ensuremath{\mathbb{e}}}
\providecommand{\Bf}{\ensuremath{\mathbb{f}}}
\providecommand{\Bg}{\ensuremath{\mathbb{g}}}
\providecommand{\Bh}{\ensuremath{\mathbb{h}}}
\providecommand{\Bi}{\ensuremath{\mathbb{i}}}
\providecommand{\Bj}{\ensuremath{\mathbb{j}}}
\providecommand{\Bk}{\ensuremath{\mathbb{k}}}
\providecommand{\Bl}{\ensuremath{\mathbb{l}}}
\providecommand{\Bm}{\ensuremath{\mathbb{m}}}
\providecommand{\Bn}{\ensuremath{\mathbb{n}}}
\providecommand{\Bo}{\ensuremath{\mathbb{o}}}
\providecommand{\Bp}{\ensuremath{\mathbb{p}}}
\providecommand{\Bq}{\ensuremath{\mathbb{q}}}
\providecommand{\Br}{\ensuremath{\mathbb{r}}}
\providecommand{\Bs}{\ensuremath{\mathbb{s}}}
\providecommand{\Bt}{\ensuremath{\mathbb{t}}}
\providecommand{\Bu}{\ensuremath{\mathbb{u}}}
\providecommand{\Bv}{\ensuremath{\mathbb{v}}}
\providecommand{\Bw}{\ensuremath{\mathbb{w}}}
\providecommand{\Bx}{\ensuremath{\mathbb{x}}}
\providecommand{\By}{\ensuremath{\mathbb{y}}}
\providecommand{\Bz}{\ensuremath{\mathbb{z}}}
\providecommand{\BA}{\ensuremath{\mathbb{A}}}
\providecommand{\BB}{\ensuremath{\mathbb{B}}}
\providecommand{\BC}{\ensuremath{\mathbb{C}}}
\providecommand{\BD}{\ensuremath{\mathbb{D}}}
\providecommand{\BE}{\ensuremath{\mathbb{E}}}
\providecommand{\BF}{\ensuremath{\mathbb{F}}}
\providecommand{\BG}{\ensuremath{\mathbb{G}}}
\providecommand{\BH}{\ensuremath{\mathbb{H}}}
\providecommand{\BI}{\ensuremath{\mathbb{I}}}
\providecommand{\BJ}{\ensuremath{\mathbb{J}}}
\providecommand{\BK}{\ensuremath{\mathbb{K}}}
\providecommand{\BL}{\ensuremath{\mathbb{L}}}
\providecommand{\BM}{\ensuremath{\mathbb{M}}}
\providecommand{\BN}{\ensuremath{\mathbb{N}}}
\providecommand{\BO}{\ensuremath{\mathbb{O}}}
\providecommand{\BP}{\ensuremath{\mathbb{P}}}
\providecommand{\BQ}{\ensuremath{\mathbb{Q}}}
\providecommand{\BR}{\ensuremath{\mathbb{R}}}
\providecommand{\BS}{\ensuremath{\mathbb{S}}}
\providecommand{\BT}{\ensuremath{\mathbb{T}}}
\providecommand{\BU}{\ensuremath{\mathbb{U}}}
\providecommand{\BV}{\ensuremath{\mathbb{V}}}
\providecommand{\BW}{\ensuremath{\mathbb{W}}}
\providecommand{\BX}{\ensuremath{\mathbb{X}}}
\providecommand{\BY}{\ensuremath{\mathbb{Y}}}
\providecommand{\BZ}{\ensuremath{\mathbb{Z}}}

\usepackage[ngerman]{babel}
\renewcommand{\id}{\ensuremath{Id}}
\renewcommand{\familydefault}{\rmdefault}





\begin{document}


\section{Einleitung}
	In diesem Vortrag geht es vorallem um Syzygien und Freie Aufl\"osungen, 
	was der Title nahe legt.
	Was der Title aber nicht erf\"ullen kann, 
	ist es uns einen Grund zu geben nach diesen Dingen zu suchen.
	Fangen wir doch damit an, 
	wie eine Syzygie definiert ist um zu sehen
	, was sie ist.
	Auf dem Weg sammeln wir noch ein paar Grundlagen mit ein:
	\begin{defn}[\nocite{Eis1} ]
		Ein \textbf{Komplex von \( R \)-Modulen} 
		ist eine Sequenz von Modulen 
		\( F_{i} \)
		und Abbildungen
		\( F_{i} \to F_{i-1} \),
		sodass f\"ur alle 
		\( i \) 
		die Komposition
		\( F_{i+1} \to F_{i} \to F_{i-1} \)
		Null wird. \\
		Die \textbf{Homologie} dieses Komplexes in 
		\( F_{i} \)
		ist der  Modul
		\[
			\ker\left( F_{i} \to F_{i-1 } \right) \big/ 
			\im \left( F_{i+1} \to F_{i} \right).
		\]		
		Eine \textbf{freie Aufl\"osung} eines 
		\(R\)-Moduls 
		\(M\)
		ist ein Komplex
		\[
			\mathcal{F}: \dots \to F_{n} 
			\overset{\phi_{n}}{\to} 
			\dots \to F_{1}
			\overset{\phi_{1}}{\to} F_{0}
		\]
		von freien
		\(R\)-Modulen, sodass 
		\( \coker \phi_{1} = M \)
		und
		\( \mathcal{F} \)
		exakt ist.
		(Man schreibt auch gelegentlicht ein 
		\( \to 0 \) 
		an das Ende des Komplexes
		und fordert Exaktheit ausser in 
		\( F_{0} \).
		Diese schreibweise wird haeufig missbraucht,
		um zu sagen, 
		dass die exakte Sequenz
		\[
			\mathcal{F}: \dots \to F_{n} 
			\overset{\phi_{n}}{\to} 
			\dots \to F_{1}i
			\overset{\phi_{1}}{\to} F_{0}
			\to M 
			\to 0
		\]
		eine Aufloesung von 
		\(M \)
		ist.
		Das Bild von
		\( \phi_{i} \)
		nennen wir die
		\emph{i-te Syzygie} 
		von 
		\( M \).\\
		Ein Aufl\"osung heisst freie, graduierte Aufl\"osung,
		wenn 
		\( R \) ein graduierter Ring,
		die 
		\( F_{i} \)
		graduierte freie Module
		und die Abbildungen homogen vom Grad 0 sind.
		Wenn es ein
		\( n < \inf \) 
		gibt, sodass 
		\(F_{n+1}=0 \),
		aber 
		\( F_{i} \neq 0 \forall 0 \le i \le n \),
		nennen wir 
		\( \mathcal{F} \) 
		eine endliche Aufl\"osung von L\"ange 
		\( n \).
	\end{defn}
	{\color{red}Nur f\"ur Expose}
	Hier eine kurze Auffrischung der Begrifflichkeiten:
	\begin{defn}[Freier Modul\cite{Eis1}{0.3}]
		Ein freier 
		\(R\)-Modul
		ist ein Modulm der isomorph zu einen direkten Summe von 
		\( R \)
		Kopien ist.
	\end{defn}
	\begin{bsp}
		Ein sehr einfaches Beispiel ist 
		\[
			M:= \mathbb{R} \cdot x \oplus
			\mathbb{R} \cdot y \oplus
			\mathbb{R} \cdot z =
			\mathbb{R}^3
		\]
	\end{bsp}
	\begin{defn}[Graduierter Ring\cite{Eis1}{1.5}]
		Ein \bf{graduierter Ring} ist ein Ring 
		\( R \)
		zusammen mit einer direkten Summenzerlegung
		\[
			R=R_{0} \oplus R_{1} \oplus R_{2} \oplus \dots 
			\text{ als abelsche Gruppen,}
		\]
		sodass 
		\[
			R_{i}R_{j} \subset R_{i+1} \text{for} i,j \ge 0.
		\]
	\end{defn}
	\begin{bsp}
		Der einfachste graduierte Ring ist der Ring Polynome
		\( S = k\left[ x_{1}, \dots ,x_{r} \right] \)
		mit der Graduierung 
		\[
			S=S_{0} \oplus S_{1} \oplus \dots ,
		\]
		wobei 
		\( S_{d} \) 
		der Vektorraum der homogenen Polynome vom Grad
		\( d \) 
		ist.
	\end{bsp}
	
	Also das hat jetzt nicht so viel zum Verst\"andnis beigetragen, 
	wie ich dachte;  
	wobei,
	unsere Syzygie besteht also aus Punkten 
	aus einer direkten Summe, 
	sozusagen einem 
	\(n\)-Tupel 
	\(a_{1},\dots,a_{n}\).
	Desweiteren folgt aufgrund der Exaktheit und der Tatsache, 
	das unsere Abbildungen von Grad 0 sind,
	d.h. sie sind jeweils der Form
	\[
		f\left( x \right)=x_{1}f_{1} + \dots + x_{k}f_{k}
	\],
	dass alle Punkte der Syzygie die folgende Gleichung erf\"ullen:
	\[
		a_{1} f_{1} + \dots + a_{n} f_{n} = 0	
	\]
	ALso sind unsere Syzygien L\"osungen von linearen Gleichungen.
	
	Syzygien werden in Computeralgebra System verwendet,
	um Multivariate Gleichungen zu loesen\nocite{WA_1}

	
\section{Das Hilbertsche Syzygientheorem}
	\begin{thm}[Das Hilberstsche Syzygien Theorem \cite{Eis1}{1.13}]
		Wenn 
		\( R = l \left[ x_1,\dots,x_r \right] \)
		gilt,
		dann hat jeder endlich erzeugte graduierte 
		\(R \)-Modul 
		eine endlich erzeugte freie Aufl\"osung von L\"ange 
		\( \le r \) 
		aus endlich erzeugten freien Moduln.
	\end{thm}
	Um das jetzt aber zu beweisen m\"ussen wir etwas ausholen, 
	doch wollen wir mit einem Beispiel beginnen.
	\begin{bsp}[\nocite{Eis1}{Exercise 1.22}]
		Sei 
		\( R = k\left[ x \right] \)
		Dann haben alle endlich erzeugten 
		\(R\)-Module 
		eine endliche freie Aufl\"osung.
		\begin{proof}
			{\color{red} REPLACE ME }		
		\end{proof}
	\end{bsp}	
	\begin{thm}[Struktursatz f\"ur endlich erzeugte Moduln \"uber Hauptidealringen\cite{Wiki_1}]
		
	\end{thm}
	\begin{bsp}
		Sei 
		\( R= k\left[ x \right] \/ \left( x^{n} \right) \),
		dann sehen die freien Aufl\"osungen von 
		\( R\big/ \left( x^{m} \right) \) 
		f\"ur alle 
		\( m \le n \)
		als 
		{\color{red}MISSING}
		geschrieben werden k\"onnen.
		\begin{proof}
			{\color{red} REPLACE ME }
		\end{proof}
	\end{bsp}

section{Gr\"obner Basen}
\begin{defn}[Initialer Term \cite{Eis1}{325}]
	Wenn 
	\( > \)
	eine Ordnung auf Mononomen ist,
	dann definieren wir f\"ur alle 
	\( f \in F \)
	den \textbf{initialen Term von \( f\) },
	geschrieben als 
	\( \ini_{>} \left( f \right) \),
	als den gr\"ossten Term von 
	\( f \) 
	bez\"uglich der Ordnung
	\( > \).
	Wenn 
	\( M \) 
	ein Untermodul von 
	\( F \) 
	ist,
	dann bezeichnet 
	\( \ini_{>} \left( M \right) \)
	den Untermodul, 
	der von 
	\( \left\{ \ini_{>} \left( f \right) \big| f \in F \right\}\)
	erzeugt wird.
	Wir werden oft 
	\( \ini_{>} \)
	mit 
	\( \ini \)
	abk\"urzen,
	wenn die Ordnung klar ist.
\end{defn}
\begin{bsp}
	Sei 
	\( >_{lex} \) 
	Die Ordung in der gilt
	\( x > y \),
	dann ist 
	\( \ini_{>_lex} \left( x^{2}+xy +y^{2} \right)=x^{2} \)

\end{bsp}
\begin{defn}[Gr\"obnerbasis \cite{Eis1}{328}]
	Eine \textbf{Gr\"obnerbasis} bez\"uglich einer Ordnung 
	\( > \)
	auf einem freien Modul mit Basis
	\(F \)
	ist eine Menge von Elementen
	\( g_{1},dots,g_{t} \ini F \)
	,sodass wenn 
	\( M \) 
	ein Untermodul von 
	\( F \) 
	erzeugt von
	\( g_{1}, \dots , g_{t} \)
	ist,
	gilt, dass die
	\( \ini_{>}  \left(  g_{1} \right),
	\dots,
	\ini_{>} \left( g_{t} \right) \)
	\( \ini_{>} \left( M \right) \)
	erzeugen. \\
	Wir sagen dann,
	dass
	\( g_{1},\dots,g_{t} \)
	eine
	\textbf{Gr\"obnerbasis von \( M \)}
	sind.

\end{defn}

\begin{nota}[\cite{Eis1}{322}\label{syzygygenerationnotation}]
	Sei 
	\( F \) 
	ein freies Modul mit Basis 
	und
	\( M \)
	Untermodul von 
	\( F \)
	der von 
	\( m_{1} , \dots , m_{t} \)
	erzeugt wird.
	Sei
	\begin{align*}
		\phi:	& \oplus_{j=1}^{t} S\epsilon_{j} &\to   F \\
			& \phi\left( \epsilon_{j} \right) &=  m_{j}
	\end{align*}
	die Abbildung eines freien Moduls, dessen Bild
	\( M \)
	ist.
	F\"ur alle Indexpaare 	
	\( i,j \),
	sodass 
	\( m_{i}\)
	und 
	\( m_{j}\)
	dasselbe Basiselement von 
	\( F \),
	definieren wir 
	\[
		m_{ij} := m_{i}/\GCD\left( m_{i},m_{j} \right) 
	\],
	wobei wir die 
	\( \sigma_{ij}\)
	wie folgt als Elemente von 
	\( \ker \phi \)
	definieren:
	\[
		\sigma_{ij} := m_{ji}\epsilon_{i}-m_{ij}\epsilon_{j} .
	\]

\end{nota}

\begin{lem}[\cite{Eis1}{15.1}\label{syzygygen}]
	Mit der Notation  \ref{syzygygenerationnotation} gilt,
	dass 
	\( \ker \phi \)
	von den
	\( \sigma_{ij} \)
	erzeugt wird.
\end{lem}

\bibliography{literature}
\bibliographystyle{plain}


\end{document}
