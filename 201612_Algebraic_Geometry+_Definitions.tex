//normally you would have a begin document and a preamble here, but not today

// i am using custom defined environments for my LaTeX Session

\begin{defn}[Complex of $R$-modules \nocite[Eis1]{1.10} ]
// the layout here is begin{environment}[title]
// and i can just start typing text
// are comments as they often are

A \bold{complex} of $R$-Modules is a sequence of modules \( F_{i} \)

// it is not needed to put every low index into brackets, but it is a 
// good habit as bigger indexes may be broken up otherwise

and maps \( F_{i} to F_{i-1} \) such that the compositions \( F_{i+1} \to F_{i} \to \F_{i-1} \) are all zero.

// as you have seen i can start math mode by either using  \$ and \(  \) 
// and my LaTeX did become quite rusty since the last time i used it
// that line gives a hint on your current mode

The \bold{homology} of this complex at \( F_{i} \) is the module 
\[
	// this starts the big math mode
	\ker \left( F_{i} \to F_{i-1} \right)   \/ \im \left( F_{i+1} \to F_{i} \right)
\]

// let us not forget to put a reference, where this definition comes from
// Eis1 Eisenbud-Harris Algebraic Geometry - A second course

A \bold{free resolution} of an \( R\)-module \( M \)

// \( \) is the newer and more resilient way of starting math environments

is a complex
\[
	\mathcal{F}: \hdots \to F_{n} \overset{\to}{\phi_{n}} 
	\hdots \to F_{1} \overset{\to}{\phi_{1}} F_{0}
\]
of free \(R\)-Modules such that \( \coker \phi_{1} = M \) 
and \( \mathcal{F} \)  is exact (sometimes we add `` \( \to 0 \) '' to the right of \(\mathcal{F}\) and then insist that \(\mathcal{F}\) be exact except at \( F_{0} \) ).
We shall sometimes abuse this notation and say that an exact sequence 
\[
	\mathcal{F}: \hdots \to F_{n} \overset{\to}{\phi_{n}} 
	\hdots \to F_{1} \overset{\to}{\phi_{1}} F_{0}
	\to M \to 0
\]
is a resolution of \( M \).  
The image of the map \( \phi_i \) is called the ith syzygy module of \(M \).
A resolution \( \mathcal{F }\) is a \bold{graded free resolution } if \( R \) is a graded ring, the \(  F_{i}\) are graded free modules, and the maps are homogeneous maps of degree 0.
Of course only graded modules can have graded free resolutions.
If for some \( n < \inf \) we have \( F_{n+1}=0 \), 
but \( F_{i} \neq 0 \forall 0 \le i \le n \), then we shall say that 
\( \mathcal{F}\) is a \bold{finite resolution of length} \( n\).
\end{defn}

// wow that was a piece of work, but there are more to come yet.





// BTW even though there are linebreaks in the code, there will be no line breaks in the compiled document.
// as you may see the structure is becoming increasingly complex; therefore we should try to get these structure visable



