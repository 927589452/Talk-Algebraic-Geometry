%!TeX spellcheck = de_DE
%!TeX encoding = utf8
%%these are ``magic setting that help the compiler

%the following lines shoul resdie within a preamble file



\documentclass{article}
%preamble starts here
\usepackage[left=3cm,right=3cm]{geometry}
\usepackage[T1]{fontenc}
\usepackage{lmodern}
\usepackage{standalone}
\usepackage{standalone}
\usepackage{xspace}
\usepackage{xcolor}
\usepackage{url}

\input{./Packages_tikz}
\usepackage{amsmath}
\usepackage{amssymb}
\usepackage{amsthm}
\usepackage{nicefrac}

\newtheorem{thm}{Theorem}
%\newcounter{thm}
\newtheorem{satz}[thm]{Satz} 
\newtheorem{propdef}[thm]{Proposition-Definition}
\theoremstyle{definition}
\newtheorem{defn}{Definition}
\newtheorem{nota}{Notation}
\newtheorem{algo}{Algorithmus}
%\newtheorem{defi}[thm]{Definition} 
\newtheorem{lem}[thm]{Lemma} 
\newtheorem{beh}[thm]{Behauptung} 
\newtheorem{bsp}[thm]{Beispiel} 

%mathoperators
\DeclareMathOperator{\im}{im}
\DeclareMathOperator{\coker}{coker}
\DeclareMathOperator{\ini}{in}
\DeclareMathOperator{\GCD}{ggT}
\DeclareMathOperator{\LCM}{kgV}

\input{./Packages_bibtex}
\input{./Abbreviations}
\input{./Abbreviations_math}
\usepackage[ngerman]{babel}
\renewcommand{\id}{\ensuremath{Id}}
\renewcommand{\familydefault}{\rmdefault}

%documentspecific
\title{Freie Aufloesungen und das Syszygien Theorem}





\begin{document}
% 
%hello watchers
%say Hi :D

% normally you would have a begin document and a preamble here, but not today

% i am using custom defined environments for my LaTeX Session

\begin{defn}[Complex of \(R\)-modules \nocite{Eis1}{1.10} ]
% the layout here is begin{environment}[title]
% and i can just start typing text
% are comments as they often are

A \bf{complex} of \(R\)-Modules is a sequence of modules \( F_{i} \)

% it is not needed to put every low index into brackets, but it is a 
% good habit as bigger indexes may be broken up otherwise

and maps \( F_{i} to F_{i-1} \) such that the compositions \( F_{i+1} \to F_{i} \to F_{i-1} \) are all zero.

% as you have seen i can start math mode by either using  \$ and \(  \) 
% and my LaTeX did become quite rusty since the last time i used it
% that line gives a hint on your current mode

The \bf{homology} of this complex at \( F_{i} \) is the module 
\[
	% this starts the big math mode
	\ker \left( F_{i} \to F_{i-1} \right)   \/ \im \left( F_{i+1} \to F_{i} \right)
\]

% let us not forget to put a reference, where this definition comes from
% Eis1 Eisenbud-Harris Algebraic Geometry - A second course

A \bf{free resolution} of an \( R\)-module \( M \)

% \( \) is the newer and more resilient way of starting math environments

is a complex
\[
	\mathcal{F}: \hdots \to F_{n} \overset{\to}{\phi_{n}} 
	\hdots \to F_{1} \overset{\to}{\phi_{1}} F_{0}
\]
of free \(R\)-Modules such that \( \coker \phi_{1} = M \) 
and \( \mathcal{F} \)  is exact (sometimes we add `` \( \to 0 \) '' to the right of \(\mathcal{F}\) and then insist that \(\mathcal{F}\) be exact except at \( F_{0} \) ).
We shall sometimes abuse this notation and say that an exact sequence 
\[
	\mathcal{F}: \hdots \to F_{n} \overset{\to}{\phi_{n}} 
	\hdots \to F_{1} \overset{\to}{\phi_{1}} F_{0}
	\to M \to 0
\]
is a resolution of \( M \).  
The image of the map \( \phi_i \) is called the ith syzygy module of \(M \).
A resolution \( \mathcal{F }\) is a \bf{graded free resolution } if \( R \) is a graded ring, the \(  F_{i}\) are graded free modules, and the maps are homogeneous maps of degree 0.
Of course only graded modules can have graded free resolutions.
If for some \( n < \inf \) we have \( F_{n+1}=0 \), 
but \( F_{i} \neq 0 \forall 0 \le i \le n \), then we shall say that 
\( \mathcal{F}\) is a \bf{finite resolution of length} \( n\).
\end{defn}

% wow that was a piece of work, but there are more to come yet.





% BTW even though there are linebreaks in the code, there will be no line breaks in the compiled document.
% as you may see the structure is becoming increasingly complex; therefore we should try to get these structure visable

\begin{thm}[Hilberts syzygy theorem\nocite{Eis1}{1.13} ]
	If \( R = l \left[ x_{1}, \dots ,x_{r} \right] \), 
then every finitely generated graded \( R\)-Module has a finite graded free resolution of lenght \( \le r  \).
\end{thm}

%%%%this is then used to Proof Hilberts Theomrem (1.11) 

%chapter 15 starts on 317

\begin{thm}[Buchberger's Criterion \nocite{Eis1}{15.8}]
	The elements \( g_1,\dots,g_t \) form a Gr\"obner basis \( \iff \ h_{ij}=0 \) forall \( i\) and \(j\)
\end{thm}


%proof by construction lies at 336
\begin{nota}[\nocite{Eis1}]
	In the following we let
	\( F \) 
	be a free module with basis and let 
	\( M \)
	be a submodule of 
	\( F \) generated by monomials
	\( m_{1},\dots,m_{t}\).
	Let
	\[
		\phi:\oplus_{j=1}^{t} S\epsilon_{j} \to F;
		\phi\left( \epsilon_{j} \right) = m_{j}
	\]
	be a homomorphism from a free module whose image is 
	\(M \).
	For each pair of indices 
	\(i,j\)
	such that 
	\( m_{i}\)
	and 
	\( m_{j}\)
	involve the same basis element of
	\( F \),
	we define 
	\[
		m_{ij} := m_{i}/\GCD\left( m_{i},m_{j} \right) ,
	\]
	and we define 
	\( \sigma_{ij} \)
	to be the element of 
	\( \ker \phi \) 
	given by
	\[
		\sigma_{ij} := m_{ji}\epsilon_{i}-m_{ij}\epsilon_{j} .
	\]
\end{nota}

\begin{lem}[\cite{Eis1}[15.1]\label{syzygy_generation}]
	With notation as above 
	,
	\( \ker \phi \)
	is generated by the 
	\( \sigma_{ij}\).
\end{lem}

\begin{defn}[initial term \nocite{Eis1}{325}]
	If \( > \) is a monomial order, 
	then for any \( f \in F \) we define the \bf{initial term of f}, 
	written \( \bf{in}_{>}\left( f \right) \) 
	to be the greatest term term of \( f \) 
	with respect to the order \( > \), 
	and if \( M \) is a submodule of \( F\) 
	we define \( in_{>}\left( M \right) \) 
	to be the monomial submodule generated by 
	the elements \( in_{<} \left( f \right) \) 
	forall \( f \in M \). 
	When there is no danger of confusion 
	we will simply write \( \bf{in} \) in place of \( \bf{in}_{>} \).
\end{defn}

%the definition of the monomial orders is optional

%Lexicographic order
%Homogeneous lexicographic order
%Reverse lexicographic order


\begin{defn}[Groebnerbasis \nocite{Eis1}{328}]
	A \bf{Gr\"obner basis} with respect to an order \(  > \) on a free module with basis \( F\) is a set of elements \( g_1, \dots , g_t \in F \) such that if \( M \) is the submodule of \( F \) generated by \( g_1 , \dots,g_t \), then \( in_{>} \left(  g_1 \right),\dots, in_{>} \left( g_t \right) \) generate \( in_{>} \left( M \right)\).
	We then say, that \( g_1, \dots, g_t \) is a \bf{ Gr\"obner basis for \( M \) }
\end{defn}
\begin{propdef}[ \nocite{Eis1}{15.6}]
	Let \( F \) be a free \( S \)-module with basis and monomial order \( > \). If \( f,g_1,\dots,g_t \in F \) then there is an expression
	\[
		f= \sum f_i g_i +f' \text{with} f' \in F, f_i \in S_i ,
	\]
	where none of the monomials of \( f'\) is in 
	\( \left( in(g_1),\dots,in(g_t) \right) \) and
	\[
		in\left( f \right) \ge in\left( f_i g_i  \right)
	\]
	for every \( i \). Any such \( f' \) is called a \bf{remainder} of \( f \) with respect to \( g_1,\dots,d_t \), and an expression 
	\( f=\sum f_i g_i + f' \)
	satisfying the condition of the proposition is called a \bf{standard expression} for \( f \) in terms of \( g_i \).
\end{propdef}
\begin{algo}[Division Algorithm \nocite{Eis1}{328}]
	Lef \( F \) be a free \( S\)-module with basis and a fixed monomial order. 
	If 
	\( f,g_1,\dots,g_t \in F \)
	, then we may produce a standard expression
	\[
		f=\sum m_u g_{s_u} +f'
	\]
	for \(f\) with respect to 
	\( g_1,\dots,g_t \)
	by defining the indices \(s_u\)
	and the terms \(m_u\) inductively.
	Having chosen 
	\( s_1,\dots,s_p \)
	and 
	\( m_1,\dots,m_p\),
	if 
	\[
		f'_p:=f-\sum_{u=1}^{p} m_u g_{s_u} \neq 0
	\]
and \( m \) is the maximal term of \(f'_p \) that is divisible by some \( in\left( g_i \right) \),
then we choose
	\[
		s_{p+1}=i , \\
		m_{p+1}=m / in\left( g_i \right)
	\]
	This process terminates when either 
	\( f'_p=0 \)or no \( in\left( g_i \right) \) divides a monomial of 
	\( f'_p\);
	the remainder \(f'\) is then the last \( f'_p \) produced.
\end{algo}

\begin{algo}[Buchberger's Algorithm \nocite{Eis1}{333}]
	In the situation of Theorem 15.8, suppose that \( M \) is a submodule of \( F \), and let
	\( g_1,\dots,g_t \in M \)
	be a set of generators of \( M \) .
	Compute the remainders \( h_{ij} \).
	If all of the \( h_{ij}=0 \), then the \( g_i \) form a Gr\"obner basis for \( M \).
	If some \( h_{ij} \neq 0\),then replace 
	\( g_1, \dots,g_t \)
	with 
	\( g_1, \dots,g_t,h_{ij} \)
	, and repeat the process.
	As the submodule generated by the initial forms of
	\(g_1,\dots,g_t,h_{ij} \)
	is strictly larger than that generated by the initial forms of
	\( g_1, \dots,g_t \),
	this process must terminate after finitely many steps.
	The upperbound \[
		b=\left( \left( r+1 \right)\left( d+1 \right)+1 \right)^{2^{(s+1)}(r+1) }
	\],
	where 
	\[
		r = \text{number of variables} \\
		d = \text{maximum degree of the polynomials \( g_i\) , and } \\
		s = \text{the degree of the Hilbert polynomial ( this is one less than the dimension; it is between \( 0 \) and \(r-1\) ).}
	\]
\end{algo}
\begin{defn}[\nocite{Eis1}{334}]
	\[
		\tau_{ij}:=m_{ji}\epsilon_{i} - m_{ij} \epsilon_{j} - \sum_{u} {f_{u}}^{\left( ij \right) } \epsilon_{u} 
	\],
	for \( i < j \) such that \( in \left( g_{i}  \right) \) and \( in \left( g_{j}  \right) \)
	involve the same basis element of \( F \).
\end{defn}

\begin{thm}[Schreyer \cite{Eis1}[15.10]]
	With the notation as above, suppose that 
	\( g_1,\dots,g_t\)
	is a Gr\"obner basis.
	Let \( > \) be the monomial order on 
	\( \oplus_{j=1}^t S \epsilon_{j} \)
	defined by taking 
	\( m \epsilon_{u} > n \epsilon_{v} \)
	\(\iff\)
	\[
		\in\left(  m g_{u} \right) > \in\left( n g_{v} \right) 
		\text{with respect to the given order on} F 
	\]
	or
	\[
		\in\left( m g_{u} \right) = \in\left( n g_{v} \right) 
		\left( \text{up to a scalar} \right) \text{but} u < v.
	\]
	The 
	\( \tau_{ij} \)
	generate the syzygies on the 
	\( g_{i} \).
	In fact, the 
	\( \tau_{ij} \)
	are a Gr\"obner basis for the syzygies with respect to the order \( > \),
	and 
	\( \in\left( \tau_{ij} \right) = m_{ji}\epsilon_{i}.\)
\end{thm}
\begin{proof}
	We show first that the initial term of 
	\( \tau_{ij} \)
	is 
	\( m_{ji} \epsilon_{i} \). We have
	\[
		m_{ji} \in \left( g_{i} \right) = m_{ij} \in \left( g_{j} \right),
	\]
	and these terms are by hypothesis greater than any that appear in the
	\( f_{u}^{\left( ij \right)} g_{u} .\)
	Thus, 
	\( \in\left( \tau_{ij} \right) \)
	is either 
	\( m_{ji} \epsilon_{i} \)
	or 
	\( -m_{ij}\epsilon_{j} \)
	by the first part of the definition of 
	\( > \),
	and since 
	\( i < j \) 
	we have 
	\( m_{ji} \epsilon_{i} > m_{ij}\epsilon_{j} .\) \\
	Now we show that the 
	\( \tau_{ij} \)
	form a Gr\"obner basis. 
	Let 
	\( \tau = \sum f_{v} \epsilon_{v}\)
	be any syzygy.
	We must show that
	\( \in\left( \tau \right) \)
	is divisable by one of the 
	\( \in\left( \tau_{ij} \right)\);
	that is,
	\( \in\left( \tau \right) \)
	is a multiple of some 
	\( m_{ji} \epsilon_{i} \)
	with
	\( i<j \).
	For each index 
	\(v \),
	set 
	\(n_{v} \epsilon_{v}=\in\left( f_{v}\epsilon_{v} \right) \).
	Since these terms cannot cancel with each other,
	we have 
	\( \in\left( \sum f_{v}\epsilon_{v} \right)=n_{i}\epsilon_{i}\)
	for some
	\( i\).
	Let 
	\( \sigma=\sum'n_{v} \epsilon_{v} \)
	be the sum over all indices
	\( v \) 
	for which 
	\( n_v \in \left( g_{v} \right)=n_{i} \in\left( g_{i} \right) \)
	up to a scalar;
	all indices 
	\( v \)
	in this sum must be 
	\( \ge i \)
	because we assume that 
	\( n_{i} \epsilon_{i} \)
	is the initial term of 
	\( \tau \). \\
	Thus,
	\( \sigma \) 
	is a syzygy on the 
	\( \in\left( g_{v} \right) \)
	with 
	\( v \ge i\).
	By \ref{syzygy_generation} ,
	all such syzygies are generate by the
	\( \sigma_{uv} \)
	for
	\( u,v \ge i \),
	and the ones in which
	\( \epsilon_{i} \),
	appears are the 
	\( \sigma_{ij}\)
	for 
	\( j > i \).
	It follows that the coefficient
	\(n_{i}\)
	is in the ideal generated by the 
	\(m_{ji}\)
	for 
	\( j> i \), 
	and we are done.
\end{proof}
\input{bibtex}
\end{document}


