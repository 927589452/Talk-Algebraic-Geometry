%!TeX spellcheck = de_DE
%!TeX encoding = utf8


\documentclass{article}
%preamble starts here
\usepackage[left=3cm,right=3cm]{geometry}
\usepackage[T1]{fontenc}
\usepackage{lmodern}
\usepackage{standalone}
\usepackage{standalone}
\usepackage{xspace}
\usepackage{xcolor}
\usepackage{url}

\input{./Packages_tikz}
\usepackage{amsmath}
\usepackage{amssymb}
\usepackage{amsthm}
\usepackage{nicefrac}

\newtheorem{thm}{Theorem}
%\newcounter{thm}
\newtheorem{satz}[thm]{Satz} 
\newtheorem{propdef}[thm]{Proposition-Definition}
\theoremstyle{definition}
\newtheorem{defn}{Definition}
\newtheorem{nota}{Notation}
\newtheorem{algo}{Algorithmus}
%\newtheorem{defi}[thm]{Definition} 
\newtheorem{lem}[thm]{Lemma} 
\newtheorem{beh}[thm]{Behauptung} 
\newtheorem{bsp}[thm]{Beispiel} 

%mathoperators
\DeclareMathOperator{\im}{im}
\DeclareMathOperator{\coker}{coker}
\DeclareMathOperator{\ini}{in}
\DeclareMathOperator{\GCD}{ggT}
\DeclareMathOperator{\LCM}{kgV}

\input{./Packages_bibtex}
\input{./Abbreviations}
\input{./Abbreviations_math}
\usepackage[ngerman]{babel}
\renewcommand{\id}{\ensuremath{Id}}
\renewcommand{\familydefault}{\rmdefault}

%documentspecific
\title{Freie Aufloesungen und das Syszygien Theorem}






\begin{document}
\maketitle
\begin{thm}[Das Hilbert'sche Syzygien Theorem \cite{Eis1}{1.13}]
	Wenn 
	\( R = l \left[ x_1,\dots,x_r \right] \)
	gilt,
	dann hat jeder endlich erzeugte graduierte 
	\(R \)-Modul 
	eine endlich erzeugte freie Aufl\"osung von L\"ange 
	\( \le r \) 
	aus endlich erzeugten freien Moduln.
\end{thm}

\begin{algo}[Divisionsalgorithmus \cite{Eis1}{15.7}]
	Sei 
	\( F \)
	ein freier 
	\( S\)-Modul
	mit Basis und fester Monomordnung.
	Wenn
	\( f,g_1,\dots,g_t \in F \),
	dann k\"onnen wir einen Standard Ausdruck
	\[
	        f=\sum m_u g_{s_u} +f'
	\]
	von
	\( f \)
	bez\"uglich 
	\( g_1,\dots,g_t \)
	finden,
	indem wir die Indices
	\( s_{u} \)
	und die Terme
	\(m_{u} \)
	induktiv definieren.
	Wenn wir bereits 
	\( s_1,\dots,s_p \)
	und
	\( m_1,\dots,m_p\),
	gew\"ahlt haben,
	dann w\"ahlen wir, 
	falls 
	\[
		f'_p:=f-\sum_{u=1}^{p} m_u g_{s_u} \neq 0
	\]
	und
	\( m \) 
	der maximale Term von 
	\(f'_p \),
	der durch eines der der
	\( in\left( g_i \right) \)
	teilbar ist,
	\[
		s_{p+1}=i , \\
		m_{p+1}=m / in\left( g_i \right)
	\].
	Dieser Vorgang bricht entweder ab, wenn 
	\( f'_p=0 \) \
	oder wenn keines der
	\( in\left( g_i \right) \) 
	ein Monom aus 
	\( f'_p\)
	teilt;
	der Rest 
	\(f'\) 
	ist dann 
	\( f'_p \).
\end{algo}
\begin{nota}[\cite{Eis1}{331}\label{criterionnotation}]
	Sei
	\( F \)
	ein freier Modul \"uber
	\(S\)
	mit Basis und Monomordnung
	\( > \).
	Seien
	\( 0 \neq g_{1},\dots,g_{t} \in F\)
	und
	\( \oplus S \epsilon_{i} \)
	ein freier Modul mit Basis
	\( \left\{ \epsilon_{i} \right\} \)
	die den
	\( \left\{ g_{i} \right\}\)
	aus
	\( F \)
	\"uber die folgenden Abbildungen
	\begin{align*}
	\phi:& \oplus S \epsilon_{i} & F \\
	& \epsilon_{i} & \mapsto g_{i} 
	\end{align*}
	entsprechen. \\
	F\"ur jedes Indexpaar
	\( i, j \),
	sodass
	\( \ini \left( g_{i} \right) \)
	und
	\( \ini \left( g_{j} \right) \)
	dasselbe Basiselement von
	\( F \)
	enthalten,
	definieren wir
	\[
	m_{ij} 
	= \ini \left( g_{i} \right) 
	/ \GCD \left( \ini\left( g_{i} \right), 
	\ini \left( g_{j} \right) \right) \in S
	\]
	und setzen
	\[
	\sigma_{ij} 
	= m_{ji} \epsilon_{i} 
	- m_{ij} \epsilon_{j}
	\].
	Diese
	\( \sigma_{ij} \)
	Erzeugen die Syzygie auf den
	\( \ini \left( g_{i} \right) \)
	nach \ref{syzygygen}.
		Desweiteren w\"ahlen wir f\"ur jedes der Indexpaare einen Standardausdruck
	\[ 
	m_{ji} g_{i} -m_{ij} g_{j} = \sum f_{u}^{\left( ij \right) } g_{u} + h_{ij}
	\]
	f\"ur
	\(  m_{ji} g_{i} -m_{ij} g_{j} \)
	bez\"uglich der
	\( g_{1},\dots , g_{t} \).
	Man kann sehen,
	dass
	\( \ini \left( f_{u}^{ij} g_{u} \right) < \ini \left( m_{ji}g_{i} \right) \)
	Zur Vereinfachung setzen wir
	\( h_{ij} = 0 \),
	falls
	\( \ini\left( g_{i} \right) \)
	und
	\( \ini\left( g_{j} \right) \)
	verschiedene Basiselemente von
	\( F \)
	enthalten.
\end{nota}
\begin{thm}[Buchberger Kriterium \cite{Eis1}{15.8}\label{criterion}]
	Mit der Notation aus \ref{criterionnotation} folgt,
	dass die 
	\( g_{1},\dots,g_{t} \)
	eine Gr\"obnerbasis bilden,
	genau dann wenn 
	\( h_{ij} \)
	f\"ur alle 
	\( i \)
	und 
	\( j\).
\end{thm}

\begin{algo}[Buchberger Algorithmus \cite{Eis1}{333}]
	Unter den Vorraussetzungen aus \ref{criterion} 
	sei
	\( M \),
	das ein Untermodul von
	\( F \)
	und 
	\( g_{1},\dots,g_{t}\)
	seien  Erzeuger von 
	\( M \).
	Berechne die Reste
	\( h_{ij} \).
	Wenn alle 
	\( h_{ij} = 0 \),
	dann bilden die 
	\(g_{i} \)
	eine Gr\"obnerbasis von 
	\( M \).
	Wenn einige der 
	\( h_{ij} \neq 0 \)
	dann ersetze 
	\( g_{1},\dots,g_{t}\)
	mit 
	\( g_{1},\dots,g_{t},h_{ij} \)
	und wiederholen dann den Prozess.
	Da der von
	\( g_{1},\dots,g_{t},h_{ij} \)
	erzeugte Untermodul echt gr\"osser als der von
	\( g_{1},\dots,g_{t}\)
	erzeugte Untermodul ist, 
	und damit terminiert der Prozess nach endlich vielen Schritten.
	Die obere Schranke 
	\[
		b=\left( 
			\left( r+1 \right)
			\left( d+1 \right)+1 
		\right)^{2^{(s+1)}(r+1) }
	\]	
	h\"alt f\"ur
	\begin{align*}	
		r = &\text{number of variables} \\
		d = &\text{maximum degree of the polynomials \( g_i\) , and } \\
		s = &\text{the degree of the Hilbert polynomial } \\
		    &\text{( this is one less than the dimension; it is between \( 0 \) and \(r-1\) ).}
	\end{align*}

\end{algo}
        \begin{defn}[\cite{Eis1}{334}\label{taunotation}]
                Wir definieren
                \[
                        \tau_{ij}:=
                        m_{ji}\epsilon_{i} - 
                        m_{ij} \epsilon_{j} - 
                        \sum_{u} {f_{u}}^{\left( ij \right) } 
                        \epsilon_{u} 
                \],
                f\"ur
                \( i < j \),
                sodass
                \( \ini \left( g_{i}  \right) \)
                und
                \( \ini \left( g_{j}  \right) \)
                dasselbe Basiselement von \( F \)
                enthalten.
        \end{defn}


\begin{thm}[Schreyer \cite{Eis1}[15.10]
	Mit der Notation von \ref{taunotation}, 
	konnen wir annehmen,
	dass
	\( g_1,\dots,g_t\)
	eine Gr\"obnerbasis sind.
	Sei jetzt 
	\( > \) 
	eine Monomordnung auf 
	\( \oplus_{j=1}^t S \epsilon_{j} \),
	f\"ur die gilt
	\( m \epsilon_{u} > n \epsilon_{v} \)
	\(\iff\)
	\[
		\ini\left(  m g_{u} \right) > \ini\left( n g_{v} \right) 
		\text{bez\"uglich der Ordnung auf } F 
	\]
	oder
	\[
		\ini\left( m g_{u} \right) = \ini\left( n g_{v} \right) 
		\left( \text{bis auf Vielfachheit} \right) \text{but} u < v.
	\].
	Die
	\( \tau_{ij} \) 
	erzeugen die Syuygien auf den
	\( g_{i} \).
	Insbesondere sind die
	\( \tau_{ij} \)
	eine Gr\"obnerbasis der Syzygien bez\"uglich der Ordnung
	\( > \)
	und 
	\( \ini\left( \tau_{ij} \right) = m_{ji}\epsilon_{i}.\)
\end{thm}

\input{./bibtex}

\end{document}



